\section{Medidas de tráfico}
\label{sec:medidas_trafico}

El tráfico vehicular, o simplemente tráfico, es el fenómeno causado por el flujo de vehículos en una vía, calle o autopista. Cuando el flujo de tráfico es elevado en una zona particular, se produce la congestión del tráfico que deriva en pérdida de tiempo y consumo excesivo de combustible para los conductores \cite{litman2011smart}. Las medidas del flujo de tráfico proporcionan información acerca de su naturaleza lo cual ayuda al análisis de las características del flujo.

Las medidas se pueden clasificar principalmente en: \begin{enumerate*}[a)]
\item mediciones de cantidad, que incluyen la densidad y el volumen de tráfico; y \item mediciones de calidad, que incluye la velocidad
\end{enumerate*}. Las medidas del flujo de tráfico pueden ser: \begin{enumerate*}[a)] \item macroscópicas, que caracterizan el tráfico como un todo; o \item microscópicas, que estudian el comportamiento de vehículos individuales \end{enumerate*}. Las medidas principales para un flujo de tráfico son: \emph{velocidad}, \emph{volumen}, y \emph{densidad} \cite{may1990fundamentals}

La \emph{Velocidad} está definida por el desplazamiento por unidad de tiempo. La velocidad observada en diferentes vehículos por lo general varía de un vehículo a otro. Para representar esa variación, varios tipos de velocidad pueden ser definidos \cite{may1990fundamentals}. Los más importantes entre ellos son: 
\begin{enumerate}

\item La \emph{velocidad de circulación} es la velocidad promedio observada para un recorrido mientras el vehículo se está moviendo. No considera el tiempo durante el cual el vehículo se encuentra detenido momentáneamente.

\item La \emph{velocidad de viaje} está dada por la distancia recorrida entre dos puntos dividida por el total del tiempo utilizado por el vehículo para completar el viaje, incluyendo los tiempos de parada. 

\item La \emph{velocidad media local} está definida como el promedio de velocidad de todos los vehículos que pasan por un punto de la carretera en un periodo de tiempo determinado.

\item La \emph{velocidad media en un tramo} está definida como el promedio de velocidad de todos los vehículos que transitan una sección de la carretera durante un periodo de tiempo específico.
\end{enumerate}

El \emph{volumen} se define como el número de vehículos que pasan por una carretera, o por un carril o dirección del camino durante un intervalo de tiempo. La medida se lleva a cabo contando el número de vehículos que pasan por un punto en particular durante un tiempo definido.

La \emph{densidad} se define como el número de vehículos ocupando una longitud de la carretera o carril en un instante dado y es generalmente expresada en vehículos por kilómetro.

La medida del tráfico más relevante para la estimación del tráfico en tiempo real es la velocidad. Esta información generalmente es utilizada para reducir la congestión en los caminos ayudando a los conductores a tomar decisiones basadas en el estado de los mismos.