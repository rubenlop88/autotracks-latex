\chapter{Medidas de tráfico}
\label{cap:5}

El tráfico vehicular, o simplemente tráfico, es el fenómeno causado por el flujo de vehículos en una vía, calle o autopista. Cuando el flujo de tráfico es elevado en una zona particular, se produce la congestión del tráfico que deriva en pérdida de tiempo y consumo excesivo de combustible para los conductores \cite{litman2011smart}.

La ingeniería de tráfico estudia el comportamiento del tráfico con el objetivo de diseñar una infraestructura para un funcionamiento fluido, seguro y económico del tráfico \cite{kadiyali1987traffic}. El flujo del tráfico, al igual que el flujo del agua, tiene una gran cantidad de medidas asociadas a él. Las medidas del flujo de tráfico proporcionan información acerca de su naturaleza, que ayuda al análisis de las características del flujo. Entender el comportamiento del tráfico requiere un conocimiento profundo de las medidas del flujo de tráfico y sus relaciones entre sí.

\section{Medidas del flujo de tráfico}

El flujo de tráfico incluye la combinación del comportamiento de los conductores y vehículos. Debido a que el comportamiento de los conductores no es uniforme, la naturaleza del flujo de tráfico tampoco lo es. Es influenciada no sólo por las características individuales de los vehículos y conductores, sino que también por la forma en que interactúan entre sí. Así, en dos instantes de tiempo diferentes, el flujo de tráfico a través de una misma calle con características definidas puede ser diferente en función al comportamiento humano observado para cada instante.

La ingeniería de tráfico, para propósitos de planificación y diseño, asume que estos cambios están dentro de ciertos rangos que pueden ser predecidos \cite{papacostas1987fundamentals}. Por ejemplo, si la velocidad máxima permitida en una carretera es de 60 kilómetros por hora, se puede suponer que todo el flujo de tráfico se mueve a una velocidad promedio de 40 kilómetros por hora en vez de a 100 o 20 kilómetros por hora. 

Las medidas se pueden clasificar principalmente en: \begin{enumerate*}[a)]
\item mediciones de cantidad, que incluyen la densidad y el volumen de tráfico; y \item mediciones de calidad, que incluye la velocidad
\end{enumerate*}. Las medidas del flujo de tráfico pueden ser: \begin{enumerate*}[a)] \item macroscópicas, que caracterizan el tráfico como un todo; o \item microscópicas, que estudian el comportamiento de vehículos individuales \end{enumerate*}. Las medidas principales para un flujo de tráfico son: \emph{velocidad}, \emph{volumen}, y \emph{densidad} \cite{may1990fundamentals}

\section{Velocidad}

La velocidad es considerada una medida de calidad del viaje. Está definida por el desplazamiento por unidad de tiempo. Matemáticamente la velocidad $v$ está dada por,
\begin{equation}
v\quad =\quad \frac { d }{ t }
\end{equation}
donde, $v$ es la velocidad del vehículo en metros sobre segundos, $d$ es la distancia recorrida en metros y $t$ el tiempo en segundos. La velocidad observada en diferentes vehículos por lo general varía de un vehículo a otro. Para representar esa variación, varios tipos de velocidad pueden ser definidos. Los más importantes entre ellos son: la \emph{velocidad local o instantánea}, la \emph{velocidad de circulación}, la \emph{velocidad de viaje}, la \emph{velocidad media local} y la \emph{velocidad media en un tramo} \cite{may1990fundamentals}.

\subsection{Velocidad local}

La \emph{velocidad local} se refiere a aquella que es observada en un vehículo en una ubicación específica. La velocidad local puede ser utilizada para diseñar la geometría del camino, la ubicación y el tamaño de las señales, el diseño de las señales y determinar la velocidad segura. El análisis de accidentes, el mantenimiento de caminos, y la congestión, son apartados en la ingeniería de tráfico que utilizan la velocidad local como entrada básica.

\subsection{Velocidad de circulación}

La \emph{velocidad de circulación} se refiere a la velocidad promedio observada para un recorrido en particular mientras el vehículo se está moviendo. Se calcula dividiendo la distancia del recorrido sobre el tiempo durante el cual el vehículo estuvo en movimiento, es decir, esta velocidad no considera el tiempo durante el cual el vehículo se encuentra detenido momentáneamente. La velocidad de circulación siempre será mayor o igual a la velocidad de viaje debido a que los retrasos no se tienen en cuenta para su cálculo.

\subsection{Velocidad de viaje}

La \emph{velocidad de viaje} está dada por la distancia recorrida entre dos puntos dividida por el total del tiempo utilizado por el vehículo para completar el viaje, incluyendo los tiempos de parada. Normalmente la velocidad de viaje es inferior a la velocidad de circulación, lo que indica que el viaje sigue una condición de parada-marcha. Una uniformidad entre las velocidades de viaje y de circulación denota condiciones de viaje confortables.

\subsection{Velocidad media local}

La \emph{velocidad media local} está definida como el promedio de velocidad de todos los vehículos que pasan por un punto de la carretera en un periodo de tiempo determinado. La velocidad media local está dada por
\begin{equation}
{ v }_{ t }=\frac { 1 }{ n } \sum _{ i=1 }^{ n }{ { v }_{ i } }
\end{equation}
donde $v_{i}$ es la velocidad local del i-ésimo vehículo, y $n$ es el número de observaciones. En  muchos estudios, las velocidades son representadas en forma de tabla de frecuencia. Entonces la velocidad media local está dada por
\begin{equation}
{ v }_{ t }=\frac { \sum _{ i=1 }^{ n }{ { q }_{ i }{ v }_{ i } }  }{ \sum _{ i=1 }^{ n }{ { q }_{ i } }  } 
\end{equation}
donde $q_{i}$ es el número de vehículos que tienen la velocidad $v_{i}$, y $n$ es el número de tales categorías de velocidad.

\subsection{Velocidad media en un tramo}

La \emph{velocidad media en un tramo} está definida como el promedio de velocidad de todos los vehículos que transitan una sección de la carretera durante un periodo de tiempo específico. Siendo $v_{i}$ la velocidad local del $i$-ésimo vehículo y considerando la \emph{longitud del camino} como una unidad constante, el tiempo $t_{i}$ que le toma al $i$-ésimo vehículo completar la longitud del camino es equivalente a $\frac { 1 }{ { v }_{ i } }$. Entonces dados $n$ vehículos, el tiempo promedio de viaje $t_s$ está dado por
\begin{equation}
{ t }_{ s }=\frac { \sum _{ i=1 }^{ n }{ { t }_{ i } }  }{ n } =\frac { 1 }{ n } \sum { \frac { 1 }{ { v }_{ i } }  }
\end{equation}

A partir de la ecuación anterior, la velocidad media en un tramo ${ v }_{ s }$ está dada por
\begin{equation}
{ v }_{ s }=\frac { 1 }{ { t }_{ s } }=\frac { n }{ \sum _{ i=1 }^{ n }{ \frac { 1 }{ { v }_{ i } }  }  }
\end{equation}

Si la velocidad local está expresada como tabla de frecuencia, entonces
\begin{equation}
{ v }_{ s }=\frac { \sum _{ i=1 }^{ n }{ { q }_{ i } }  }{ \sum _{ i=1 }^{ n }{ \frac { { q }_{ i } }{ { v }_{ i } }  }  } 
\end{equation}
donde $q_{i}$ vehículos tienen la velocidad $v_{i}$ y $n$ es el número de tales observaciones.

\section{Volumen}

El \emph{volumen} se define como el número de vehículos que pasan por una carretera, o por un carril o dirección del camino durante un intervalo de tiempo. La medida se lleva a cabo contando el número de vehículos $n_{t}$ que pasan por un punto en particular durante un tiempo definido $t$. Entonces el flujo $q$ expresado en vehículos por unidad de tiempo está dado por
\begin{equation}
q=\frac { { n }_{ t } }{ t }
\end{equation}

El estudio del volumen se utiliza para establecer la importancia de una ruta en particular con respecto a otras, lo que ayuda a determinar el diseño de la carretera y de las instalaciones relacionadas a la misma. Así, el volumen puede ser considerado como la medida más importante del flujo de tráfico para el diseño de carreteras.

\section{Densidad}

La \emph{densidad} se define como el número de vehículos ocupando una longitud de la carretera o carril en un instante dado y es generalmente expresada en vehículos por kilómetro. Así, para un número de vehículos $n_{x}$ observados en una longitud de camino $x$, la densidad $k$ está dada por
\begin{equation}
k=\frac { { n }_{ x } }{ x }
\end{equation}
La densidad puede ser considerada igual de importante que el volumen, ya que está directamente relacionada con la demanda de tráfico. Normalmente es aplicada para estimar la proximidad entre los vehículos, que puede ser tenida en cuanta para mejorar aspectos tales como libertad de maniobra en una carretera.

\section{Estimación de tráfico en tiempo real}

La información de tráfico en tiempo real es esencial para dar soporte al desarrollo de aplicaciones utilizadas para la detección de incidencias, el control de semáforos inteligentes y la navegación en vehículos. Esta información generalmente es utilizada para reducir la congestión en los caminos ayudando a los conductores a tomar decisiones basadas en el estado de los mismos. La medida del tráfico más relevante para la estimación del tráfico en tiempo real es la velocidad.

Para la estimación de tráfico en tiempo real, los países desarrollados utilizan redes de sensores como fuente de datos \cite{leduc2008road}. Por ejemplo, la Dirección General de Tráfico del Ministerio del Interior de España provee datos de tráfico en tiempo real que son  integrados a los mapas de Google utilizando sensores de tráfico localizados a lo largo de la red de caminos. Esta herramienta permite recolectar el flujo de tráfico por hora y la velocidad promedio en los alrededores de Madrid. Otros países como Francia, Reino Unido y Portugal cuentan con sistemas similares.

En países en vías de desarrollo, ante la ausencia de sensores detectores se aplican soluciones basadas en FCD. En \cite{herrera2010evaluation} se presenta un experimento que utiliza teléfonos celulares equipados con GPS para obtener datos de la velocidad del tráfico en tiempo real. Este trabajo en base a sus  resultados obtenidos estima que una penetración de 2 a 3\% de teléfonos móviles entre los conductores es suficiente para proveer medidas precisas de la velocidad del flujo de tráfico. En los trabajos  \cite{reinthaler2007evaluation, sevlian2010travel,li2007practical} se utilizan taxis como vehículos de prueba para la obtención de FCD, bajo el supuesto de que las medidas de velocidad provenientes de los taxis son representativas de la velocidad promedio de toda la población de vehículos \cite{linauer2004fleet}. 

También es posible realizar predicciones del estado del tráfico basándose en la información actual e histórica del tráfico. En \cite{de2008traffic} se describe un sistema basado en FCD con vehículos que envían su ubicación cada tres minutos, y que a través algoritmos basados en Redes Neuronales y MM derivan predicciones a corto plazo del estado de tráfico a futuro, típicamente de 15 a 30 minutos. Dicho trabajo reporta un error situado entre 2\% a 8\% en predicciones de 15 minutos y entre 3\% a 16\% para predicciones de 30 minutos.